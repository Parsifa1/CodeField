\documentclass{article}
%\usepackage{ctex} 
\usepackage{lmodern}
\usepackage[hidelinks]{hyperref}
\usepackage{amsmath,amssymb,amsthm}
\usepackage{newtxmath}

\title{Li's first \LaTeX\enspace document}                  
\author{\emph{aldric li}}

\begin{document}
\maketitle
{\subsection*{Summary Sheet}}
{\large 	Forest is the largest carbon pool in the terrestrial ecosystem, which plays a very important and unique role
    in reducing the concentration of greenhouse gases in the atmosphere and slowing down global warming.
    The expansion of forest cover is an important mitigation measure that is economically feasible and less costly in the future.
    To study the carbon sequestration capacity of forests and its economic value,
    we established two m6dels: Model I, the best carbon sequestration rate model based on the Logistic Growth Model and BEF;
    Model II is the best economic benefit model based on the Logistic Growth Model.

    In Model I, we solve the problem of how to predict the future amount of carbon
    sequestration in forests and their products. To predict the amount of carbon
    sequestration in the forest, we first predict the future growth and development
    of the forest according to the Logistic Growth Model and obtain the change
    curve of the number of timber trees in the forest with time. And then calculate
    that specific stand volume of the forest in combination with the species of the
    tree which is in the forest. Then we calculate the total biomass in the forest
    according to the BEF of different forests and calculate the specific carbon
    content in the forest. According to the curve of forest carbon sequestration
    over time, we obtained the forest size under the maximum carbon sequestration
    rate.

    In Model II, we solve the problem of how to determine the best time and amount
    of logging in the case of considering economic benefits. After considering many
    factors such as inflation coefficient, logger's salary, logging efficiency,
    timber market unit price, and so on, we calculate the best logging time and the
    best number of loggers according to the Logistic growth model and the method of
    differential equation. We obtained the forest size under the best economic
    benefit of the forest was determined.

    In addition, we also predict the carbon sequestration of forests after 100
    years and propose the best strategy for cutting down trees. And wrote a news
    report to educate the public about forest management.}

~\\~\\~\\

\tableofcontents
\section{introduction}
this is my first \LaTeX document ,i am a student of harbin engineering university.
i will study \LaTeX in my best.\TeX\TeX\TeX
Wht the auto save function of \LaTeX is not working?
i set the save function key of \LaTeX to f5, and it works well in my computer.
but when i use the computer in the lab, it does not work.
In Model I, we solve the problem of how to predict the future amount of carbon
sequestration in forests and their products. To predict the amount of carbon
sequestration in the forest, we first predict the future growth and development
of the forest according to the Logistic Growth Model and obtain the change
curve of the number of timber trees in the forest with time. And then calculate
that specific stand volume of the forest in combination with the species of the
tree which is in the forest. Then we calculate the total biomass in the forest
according to the BEF of different forests and calculate the specific carbon
content in the forest. According to the curve of forest carbon sequestration
over time, we obtained the forest size under the maximum carbon sequestration
rate.

In Model II, we solve the problem of how to determine the best time and amount
of logging in the case of considering economic benefits. After considering many
factors such as inflation coefficient, logger's salary, logging efficiency,
timber market unit price, and so on, we calculate the best logging time and the
best number of loggers according to the Logistic growth model and the method of
differential equation. We obtained the forest size under the best economic
benefit of the forest was determined.

In addition, we also predict the carbon sequestration of forests after 100
years and propose the best strategy for cutting down trees. And wrote a news
report to educate the public about forest management.\TeX\TeX\TeX\TeX
\newpage
\section{main}
\subsection{start}
\subsubsection{end}
\end{document}













